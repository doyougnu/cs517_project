\label{section:introduction}%

The minimum feedback arc set problem is a canonical NP-Complete problem given
by\todo{add citation} \citet{KarpNPComplete}. Worse still, the problem is
APX-hard\todo{cite Viggo Kann thesis} yet finding the minimum feedback arc set
of a directed graph is desirable for many domains: such as certain rank-choice
voting systems, tournament ranking systems\todo{cite}, and dependency graphs in
general.

Solutions to the minimum feedback arc set problem are commonly implemented in
widely used graph libraries yet all suffer from a distinct flaw. While each
implementation provides a solution, the implementations only do so without
allowing the user to inspect intermediate steps; which might contain useful
domain information. For example, the graph \autoref{fig:tournament-graph}
displays a directed graph which encodes a single win tournament system. Each
vertex is a player and each edge encodes a win over a contestant, for example we
see that player E beat contestant B. Observe that the
\autoref{fig:tournament-graph} contains cycles which implies that there is not a
linear ordering amongst the contestants of the tournament, and so any ranking of
the contestants would violate a win of one contestant over another. Removing the
minimum feedback arc set would remove the cycles, which yields a linear order in
the tournament that violates the fewest number of wins. In this example,
intermediate results are edges which compose the minimum feedback arc set, thus
if one had access to the intermediate results one might choose to rematch the
opponents rather than remove the win. Crucially, the result of the rematch may
change the final minimum feedback arc set. Such a procedure could therefore
increase the confidence in the results of the tournament.

To allow for \emph{inspectable incrementality} we propose a novel\todo{but is it
  novel?} direction in solving the minimum feedback arc set problem based on
recent advances in \ac{sat} and \ac{smt} solving. Our approach is to utilize an
\ac{smt} solver to detect cycles and minimize the weight of the feedback arc
set. Incrementality in this approach is given through use of an
\emph{incremental} \ac{smt} solver and generation of \emph{unsatisfiable cores}.
A minimum unsatisfiable core is the minimum set of clauses in a \ac{sat} or
\ac{smt} formula which prevent a \ac{sat} or \ac{smt} solver run from finding a
satisfiable assignment. An incremental solver provides the end-user the ability
to add or remove constraints and thereby direct the solver during runtime.
Incrementality in our approach is crucial as it the end-user decision points to
interact with the solution process. Thus, an end-user might observe a
unsatisfiable core which corresponds to a cycle and decide to resample
\emph{only} those edges.

The approach has benefits \ldots{}

We make the following contributions:
\begin{enumerate}
\item Gadgets
\item Evaluation of unsatisfiable cores for this problem domain
\item empirical evaluation of this novel\todo{again, is it novel?} method
\end{enumerate}



%%% Local Variables:
%%% mode: latex
%%% TeX-master: "../main"
%%% End: