Source code listings

\begin{lstlisting}[language=python]
  """- Module    : app.py
  - Copyright : (c) Jeffrey M. Young
  ; (c) Colin Shea-Blymyer
  - License   : BSD3
  - Maintainer: youngjef@oregonstate.edu, sheablyc@oregonstate.edu
  - Stability : experimental

  """

  from pprint     import pprint
  from functools  import reduce
  import z3       as z
  import igraph   as ig
  import numpy    as np

  import matplotlib.pyplot as plt
  import timeit
  from tqdm import tqdm

  import graphs  as gs
  import utils   as u

def MFAS_set_cover(s,graph):
    """Find the minimum feedback arc set by encoding it as a minimum set cover.
    The encoding requires a cycle matrix which we find externally to the SAT
    solver. Then given the cycle matrix we do the following encoding:

    Variables:
      - m is |E|
      - w_{j} is the weight of edge j \in E (we don't implement the weight matrix)
      - y_{j} is a symbolic edge; y_{j} = 1 if edge j is in the feedback edge set and 0 otherwise
      - a is the cycle matrix
      - a_{ij} is the value of edge j in cycle i; a_{ij} = 1 if j participates, 0 otherwise


    minimize(sum{j = 1}^{m}(w_{j} * y_{j}))

      subject to:

    sum_{j = 1}^{m}(a_{ij} * y_{j}) >= 1
    \forall i. y_{i} \in {0,1}
    """

    ## initialization
    m            = graph.ecount()
    cycle_matrix = u.mk_cycle_matrix(u.find_all_cycles(graph), m)
    n, c         = graph.get_adjacency().shape
    num_cycles   = len(cycle_matrix)
    edge_list    = graph.get_edgelist()
    sym_to_edge_cache = {}
    edge_to_sym_cache = {}
    sum_var      = 'y'


    def symbolize(i,j):
        "given two indices, create a symbolic variable"
        new = z.Int('{0}->{1}'.format(i,j))
        return new


    def constraint_1(i,s_edge):
        """ Multiply the edge by its corresponding value in the cycle matrix
        """
        edge  = sym_to_edge_cache[s_edge]
        value = 0
        if edge in cycle_matrix[i]:
            value = cycle_matrix[i][edge]

        return (value * s_edge)


    ## symbolize the edges
    for source,sink in edge_list:
            s_edge                           = symbolize(source, sink)
            ## an edge is either a 0 or a 1
            s.add(z.Or([s_edge == 0, s_edge == 1]))

            sym_to_edge_cache[s_edge]        = (source,sink)
            edge_to_sym_cache[(source,sink)] = s_edge


    ## Perform constraint 1 and add it to the solver instance
    for i in range(num_cycles):
        s.add(z.Sum([constraint_1(i,s_edge)
                     for s_edge in sym_to_edge_cache.keys()]) >= 1)


    ## we want the smallest y possible
    s.minimize(z.Sum([s_edge for s_edge in sym_to_edge_cache.keys()]))

    s.check()
    return s.model()
\end{lstlisting}


\newpage
\begin{lstlisting}[language=python]
  def runWithGraph(graph):
  s = z.Optimize()
  return MFAS_set_cover(s, graph), u.get_feedback_arc_set(graph)


  def runErdosRenyi(n,p):
  """Given n vertices and a probability, p of edges. Find the minimum
  feedback arc set of an erdos-renyi graph

  """
  s = z.Optimize()
  g = ig.Graph.Erdos_Renyi(n, p, directed=True, loops=True)
  while g.is_dag():
  g = ig.Graph.Erdos_Renyi(n, p, directed=True, loops=True)

  return MFAS_set_cover(s,g), u.get_feedback_arc_set(g)

  def runWattsStrogatz(dim, size, nei, p):
  """Given the dimension of the lattice, size of the lattice along all
  dimensions, the number of steps within which two vertices are connected
  (nei), and the probability p, find the minimum feedback arc set of a
  watts-strogatz graph

  """
  s = z.Optimize()
  g = ig.Graph.Watts_Strogatz(dim, size, nei, p, loops=True, multiple=False)
  while g.is_dag():
  g = ig.Graph.Watts_Strogatz(dim, size, nei, p, loops=True, multiple=False)

  return MFAS_set_cover(s,g), u.get_feedback_arc_set(g)
\end{lstlisting}

\newpage
\begin{lstlisting}[language=python]
"""
- Module    : utils.py
- Copyright : (c) Jeffrey M. Young
            ; (c) Colin Shea-Blymyer
- License   : BSD3
- Maintainer: youngjef@oregonstate.edu, sheablyc@oregonstate.edu
- Stability : experimental

Common utility functions
"""

from z3 import *
import networkx as nx
from numpy import empty
from re import split

def make_name(frm,to): return frm + "->" + to

def parse_edge(edge): # return list(map(int,edge.__str__().split("->")))
    str_edge = edge.__str__()
    inner    = list(map(lambda x: x, str_edge.split("->")))
    return tuple(map(lambda x : int(x), inner))

def parse_core(core):
    """Parse an unsat core. An unsat core is shallowly embedded as a list of z3
    BoolRef objects such as: [1->2, 2->3, 3->4, 4->1], to operate on these we
    need to coerce them to a string parse the string and coerce the Ints out.

    Input:  List of strings, e.g.,      [1->2  , 2->3  , 3->4  , 4->1  ]
    Output: List of List of Ints, e.g., [[1, 2], [2, 3], [3, 4], [4, 1]]

    """
    return list(map(lambda e: parse_edge(e), core))

def edge_to_list_dict(g):
    """ Convert a graph to a dictionary of edges
    """
    ret = {}
    for source,sink in g.get_edgelist():
        if source not in ret.keys():
            ret[source] = []

        ret[source].append(sink)

    return ret

def remove_edge(g,source,sink):
    """Given an igraph graph, a source vertex and a sink vertex remove the edge
    connecting the source and the sink from the igraph graph. This function
    mutates g.
    """
    g.delete_edges(g.get_eid(source,sink))


def flatten(list_o_lists):
    return [e for sublist in list_o_lists for e in sublist]


def find_all_cycles(graph):
    nx_graph = graph.to_networkx()
    return list(nx.simple_cycles(nx_graph))


def pairs(ls, n = 1):
    return list(zip(ls, ls[n:] + ls[:n]))


def mk_cycle_matrix(cycle_edge_list, num_edges):
    ps          = [pairs(edges) for edges in cycle_edge_list]
    cycle_count = len(cycle_edge_list)
    matrix      = []
    for i in range(cycle_count):
        cycle = {}
        for pair in ps[i]:
            cycle[pair] = 1

        matrix.append(cycle)

    return matrix

def get_feedback_arc_set(graph):
    fas = graph.feedback_arc_set(method="ip")
    return list(map(lambda x : graph.es[x].tuple, fas))
\end{lstlisting}
