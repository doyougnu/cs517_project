\label{section:incremental}

Our original conception for this study was to explore the use of incremental
\ac{sat} and \ac{smt} solvers to solve the minimum feedback arc set problem.
However, we found fundamental problems with this approach during several courses
of experimentation. In this section, we review the problematic nature of the
incremental approach thereby enriching the problem. We provide a working
non-incremental \ac{smt} encoding in the next section.

The incremental approach was a combination of two \ac{sat} and \ac{smt}
features. First, we desired to encode the problem in such a way that an
\rn{unsat} would be returned from the solver. With an \rn{unsat} returned, we
could then query for an \emph{unsatisfiable core}, that is, the set of
constraints which prevent the solver from unifying. Second, we sought to use
\emph{scopes} to add and remove clauses from the incremental solver via
\rn{push} and \rn{pop} calls. With scopes it is possible to refine the
constraints \emph{during} the solving routine, possibly simplifying the problem
space. We hypothesized that by using scopes and unsatisfiable cores we could
reveal domain information to the user and direct the solver to a solution in an
efficient manner.

Unfortunately the incremental approach has numerous problems for finding the
minimum feedback arc set, which appear to be intractable. First, in order to
generate unsatisfiable cores constraints must be added to the solver instance
\emph{and watched}. In and of itself, this is not an intractable problem, the
\acl{smtlib} standard defines a function to track constraints called
\rn{assert\_and\_track} which takes a constraint and a name. The name is then
returned in the unsatisfiable core if the concomitant constraint is in the core.
The issue becomes a constant time cost incurred by parsing the unsatisfiable
core into a usable format for the solving routine to continue. The pattern
becomes: query for an unsatisfiable core, parse the core, refine the
constraints, and then repeat. Thus the constant time cost is exacerbated for
every core that is generated.

The more serious issue is the use of scopes is problematic for tackling the
minimum feedback arc set problem at all. While scopes would allow one to direct
the solver to a solution, or explore possible solutions, it requires knowledge
of the problem domain which is not readily available during the encoding
process. In order to employ scopes, one must know before the encoding step which
constraints are likely to change. This knowledge is required so that the
constraints can be pushed onto the assertion stack \emph{as late as possible}.
When these constraints are on top of the assertion stack (and thus pushed last)
they can be removed from the assertion stack \emph{without} removing constraints
that we do not wish to refine. Without this knowledge, there is a substantial
risk that the we may need to remove all, or almost all constraints from the
solver just to repack the solver and refine the query. In the worst case, this
requires an additional complete traversal of the graph. The issue is further
exacerbated by the minimum feedback arc set problem because it is not possible
to know before hand which edges, and thus which constraints, will need to be
refined. In fact finding the unique set of constraints that will need to be
refined is itself an instance of finding the minimum feedback arc set!


%%% Local Variables:
%%% mode: latex
%%% TeX-master: "../main"
%%% End: