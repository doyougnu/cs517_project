\label{section:case-studies}
%
To evaluate this approach, we construct a prototype \ac{smt}-enabled algorithm
and assess the prototype on Erd\H{o}s-R\'{e}nyi graphs. Erd\H{o}s-R\'{e}nyi
graphs are generated according to two parameters: $\kf{p}$, a metric of
connectedness for the graph, and $\kf{k}$ is the number of vertices in the
graph.

With these parameters we employ random generation to construct sample graphs. We
are interested in the individual effect $\kf{k}$ and $\kf{p}$ have on runtime,
in addition to the interactions effects between each parameter. Consider the
case where $\kf{p}$ and $\kf{k}$ are left unbound, yet $\kf{s}$ is set to 1.
This is the specific case where solving the minimum weighted feedback arc set
problem solves the minimum feedback arc set problem. Thus, by setting $\kf{s}$
to 1 we generate graphs to solve the minimum feedback arc set problem. Consider
the case where $\kf{s}$ is larger than one. In this case, edges must possess
positive weights and the weighted minimum feedback arc set may be different than
the minimum feedback arc set.

We provide the complete prototype implementation in
\autoref{appendix:source-code-listings}. To ensure correctness we compared
results from the \ac{smt} routine to a built in method in the python library
\lstinline{igraph} which finds the minimum feedback arc set. We observed no
differences between both methods.


%%% Local Variables:
%%% mode: latex
%%% TeX-master: "../main"
%%% End: