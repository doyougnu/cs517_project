\label{section:case-studies}

To evaluate this approach, we construct a prototype \ac{smt}-enabled algorithm
and assess the prototype on Erd\H{o}s-R\'{e}nyi graphs. Erd\H{o}s-R\'{e}nyi
graphs are generated according to two parameters: $\kf{p}$, a metric of
connectedness for the graph, and $\kf{k}$ is the number of vertices in the
graph. In addition, we add a parameter $\kf{s}$ which represents the size of the
range of possible weights on edges in a graph.

With these parameters we employ random generation to construct sample graphs. We
are interested in the individual effect $\kf{k}$, $\kf{p}$ and $\kf{s}$ have on
runtime, in addition to the interactions effects between each parameter.
Consider the case where $\kf{p}$ and $\kf{k}$ are left unbound, yet $\kf{s}$ is
set to 1. This is the specific case where solving the minimum weighted feedback
arc set problem solves the minimum feedback arc set problem. Thus, by setting
$\kf{s}$ to 1 we generate graphs to solve the minimum feedback arc set problem.
Consider the case where $\kf{s}$ is larger than one. In this case, edges must
possess positive weights and the weighted minimum feedback arc set may be
different than the minimum feedback arc set.

\NOTE{Hi Mike. The rest of the section is describing each case individually and
  then research questions. We don't have these spelled out right now. What we
  hypothesize is that at some combination of $\kf{k}$, $\kf{p}$ and $\kf{s}$ the
solver runtime will become infeasible. Then we can make determinations on the
viability of this method in different domains. For example, most tournaments
will be a bounded number of participants, and score ranges, but may have maximum
connectivity. Thus, we would be able to make a conclusion on the viability of
our method for the tournament domain.}



%%% Local Variables:
%%% mode: latex
%%% TeX-master: "../main"
%%% End: