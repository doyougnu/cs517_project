The minimum feedback arc set problem is the problem of finding the least amount
of edges in a directed, cyclic, graph, such that if these edges are removed a
directed, acyclic graph results. There are exists several classic and
approximate methods to solve the minimum feedback arc set problem. In this
study, we explore the novel use of \ac{smt} solvers to solve the minimum
feedback arc set problem. With an \ac{smt} solver, we hypothesize that not only
is an effective encoding possible, but by employing advanced features of
\ac{smt} solvers, such as incrementality and unsatisfiable cores, we can provide
users with decision points to give users the ability to pause the solution
routine, resample particular edges, alter the problem or inspect intermediate
results. Our results are negative. We find an effective \ac{smt} encoding for
the minimum feedback arc set problem but the encoding produces \ac{smt} problems
that in practice are unreasonably slow. Similarly, we find that to make
effective use of incrementality and unsatisfiable cores, one would require
information which is itself an instance of the minimum feedback arc set, and
thus these features are ill suited to this problem domain.


%%% Local Variables:
%%% mode: latex
%%% TeX-master: "../main"
%%% End: