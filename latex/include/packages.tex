%%%%%%%%%%%%%%%%%%%%%%%%%%%%%%% Packages %%%%%%%%%%%%%%%%%%%%%%%%%%%%%%%%%%%%%%%
\usepackage{graphicx}
\usepackage{epsfig}
\usepackage{times}
\usepackage[ruled,vlined]{algorithm2e}            %% for algorithms
\usepackage{amsmath}
\usepackage{amsthm}                  %% for theorems and definitions
% \usepackage{amssymb}                 %% for arrows like rightarrowtail
\usepackage{amsfonts}                %% \mathbb
\usepackage{mathtools}               %% for coloneqq and others
\usepackage{cases}                   %% better math brackets
\usepackage{mathpartir}              %%% for inference rules
\usepackage[nolist]{acronym}
\usepackage{hyperref}                %% for auto references
\usepackage{xpunctuate}              %% for punctuation's after macros
% \usepackage{cite}                    %% for bibliography ranges
\usepackage{subcaption}              %% For complex figures with subfigures/subcaptions
\usepackage{wrapfig}                 %% wrapping figures around text
\usepackage{listings}                %% for source code lstlisting
\usepackage{enumitem}                %% For better enumerate environment
\setlist[description]{leftmargin=\parindent,labelindent=\parindent}
\usepackage{caption}
\usepackage{multirow}                %% for multiple rows in tables
\usepackage{tikz}                    %% assertion stack visualization
\usepackage[T1]{fontenc}             %% For escaping special characters in bibtex
\usetikzlibrary{matrix}
\usetikzlibrary{arrows}
\usetikzlibrary{positioning}
\usetikzlibrary{shapes.multipart}
\usetikzlibrary{automata}
\usetikzlibrary{cd}                  %% for commuting diagrams

\usepackage{lib/paperCommands}

%%%%%%%%%%%%%%%%%%%%%%%%%%%%%%% End Packages %%%%%%%%%%%%%%%%%%%%%%%%%%%%%%%%%%


%%%%%%%%%%%%%%%%%%%%%%%%%%%%%%% Packages Configs %%%%%%%%%%%%%%%%%%%%%%%%%%%%%%%%%%
\graphicspath{ {./Figures/} }
\raggedbottom

\theoremstyle{definition}
\newtheorem{theorem}{Theorem}[section]
\newtheorem{definition}{Definition}[section]
\newtheorem{corollary}{Corollary}[theorem]
\newtheorem{lemma}[theorem]{Lemma}
\newtheorem*{analysis}{Analysis}

\DeclareMathOperator*{\minimize}{minimize}

\SetKwInOut{Input}{Input}                % Set the Input
\SetKwInOut{Output}{Output}              % set the Output

\lstdefinelanguage{SMTLIB}{
    alsoletter={\-},
    morekeywords={push,pop,assert,check-sat,declare,const,define,fun,and,or,not,check,sat,get,model,reset},
    morecomment=[l]{;},
    morestring=[b]{"},
    sensitive=false,
}

\lstset{
  frame=top,frame=bottom,
  basicstyle=\linespread{.99}\small\normalfont\sffamily,    % the size of the fonts that are used for the code
  commentstyle=\color{Gray},
  keywordstyle=\color{NavyBlue},
  stepnumber=1,                           % the step between two line-numbers. If it is 1 each line will be numbered
  numbersep=10pt,                         % how far the line-numbers are from the code
  tabsize=2,                              % tab size in blank spaces
  extendedchars=true,                     %
  breaklines=true,                        % sets automatic line breaking
  captionpos=t,                           % sets the caption-position to top
  mathescape=true,
  showstringspaces=false,
  escapeinside={(*@}{@*)},%
  literate = {-}{-}1
 }

\newcommand{\lstvdots}{%
  \raisebox{-1pt}[0pt][0pt]{%
    \scalebox{0.7}{\ensuremath{\vdots}}}%
  \hspace{-1.5pt}}

\reversemarginpar
%%%% new environment for sub-proofs
\newenvironment{subproof}[1][\proofname]{%
  \renewcommand{\qedsymbol}{\ensuremath{\blacksquare}}%
  \begin{proof}[#1]%
}{%
  \end{proof}%
}

\newenvironment{MAlgorithm}[1][htb]
  {\renewcommand{\algorithmcfname}{M}% Update algorithm name
   \begin{algorithm}[#1]%
  }{\end{algorithm}}

\renewcommand{\sectionautorefname}{Section }
%%%%%%%%%%%%%%%%%%%%%%%%%%%%%%% End Packages Configs %%%%%%%%%%%%%%%%%%%%%%%%%%%%%%